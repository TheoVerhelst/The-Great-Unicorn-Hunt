\documentclass[a4paper]{article}

\usepackage[T1]{fontenc}
\usepackage[margin=2.5cm]{geometry}
\usepackage{abstract}
\usepackage{url}
\usepackage{graphicx}

\author{The Unicorn Hunters\\
Christian Frantzen, Guilherme Barreiro Vieira,\\
Mortimer Sotom and Théo Verhelst}
\title{Advanced Machine Learning Coursework: Kaggle Competition NYC Taxis}

\begin{document}
\maketitle


\begin{abstract}
Accurate prediction of taxi trip duration is an important part of efficient
customer service for taxi companies. To this end, the New York City Taxi
and Limousine Commission organised a competition on the Kaggle website, aiming
to stimulate research in taxi trip duration regression. This report presents the
work that our team conducted on this problem. A classical machine learning
pipeline has been applied: data analysis, data cleaning, feature selection,
model evaluation and optimisation. The models that have been tried range from
Ridge Regression to Extreme Gradient Boosting. An unlabeled test set is proposed
to the competitors in order to accurately evaluate the generalisation
performance of their models. The predictions of our best models on this test set
gave us a decent ranking in the top 20\% of the competition leaderboard.
\end{abstract}

\section{Introduction}
New York City (NYC) is one of the busiest cities in the world with pedestrians
constantly rushing to arrive to their destination on time. To avoid searching
for parking and the stress of driving during rush hour, most citizens of NYC use
public transport, and especially taxis. The most important piece of information
is undoubtedly the estimated trip duration for customers to plan their schedule,
for the driver to find the shortest route and for deriving the estimated price
of the journey. This is exactly the aim of the ``New York City Taxi Trip
Duration'' Kaggle competition, build a model able to predict the total ride
duration of taxi trips given a set of features such as geo-coordinates, number
of passengers and others. The dataset provided is one released by the NYC Taxi
and Limousine Commission (TLC) and contains over 1.45 million trip records over
the entire year of 2016. Participants must then predict the trip duration in
seconds for each entry of the test set containing 625,134 trip records.

Before diving into building predictive models, the data must be carefully
analysed to gain an understanding of the features provided and the problem at
hand. The next was preprocessing the data and augmenting it with hourly
rainfall, further distance measurements and road related informations from The
Open Source Routing Machine (ORSM). Section four explains how the relevance of
each feature was evaluated by information gain and recursive feature selection.
The following section describes the chosen evaluation metric, Root Mean Squared
Logarithmic Error (RMSLE), and the reasoning behind also predicting the trip
durations in minutes by classification. Finally, a total of seven models were
implemented: Adaboost, Support Vector Machine (SVM), Gaussian Process,
Stochastic Gradient Descent, Random Forest, Extreme Gradient Boosting (XGBoost)
and Neural Network (NN). The results of each model is presented and discussed in
the last section of this report.


\section{Data Analysis}
The first step of a machine learning project is to explore and analyse the data,
in order to better understand the problem. Our dataset is composed of 11
features: a unique identifier for each row, the identifier of the taxi company,
the pick-up time, the pick-up and drop-off locations, the number of passengers,
and a boolean flag indicated if the trip data has been stored on-board or
directly sent to the data server.

Figures \ref{long} to \ref{pickup_hour} show the distribution of some of these
features. Pick-up month, minute and seconds are not shown, as their graphs
are uniformly distributed and therefore not visually informative. Outliers have
been removed in order to properly display the pick-up and drop-off location
distribution, as well as the trip duration. These outliers will be discussed in
the next section. One can see in figure \ref{long} and \ref{lat} that the
location seems to be roughly normally distributed, and the logarithm of the trip
duration also appears to be normally distributed in figure
\ref{log_trip_duration}. The smaller bumps outside of the main bell in the
location curves correspond to trips to the two city airports.

\begin{figure}
    \centering
   \begin{minipage}{.45\textwidth}
        \includegraphics[width=\linewidth]{longitude}
        \caption{Distribtion of the longitude.}
        \label{long}
    \end{minipage}
    \hspace{0.05\textwidth}
    \begin{minipage}{.45\textwidth}
        \includegraphics[width=\linewidth]{latitude}
        \caption{Distribution of the latitude.}
        \label{lat}
    \end{minipage}
\end{figure}

\begin{figure}
    \centering
    \begin{minipage}{.45\textwidth}
        \includegraphics[width=\linewidth]{map_kde}
        \caption{Heatmap of the trip locations on a map (credits: OpenStreetMap).}
        \label{heatmap}
    \end{minipage}
    \hspace{0.05\textwidth}
   \begin{minipage}{.45\textwidth}
       \includegraphics[width=\linewidth]{passenger_count}
       \caption{Distribution of the number of passengers.}
       \label{passenger_count}
    \end{minipage}
\end{figure}

\begin{figure}
    \centering
    \includegraphics[width=\linewidth]{pickup_hour_vs_trip_duration}
    \caption{Distribution of the pick-up time, for different days of the week.}
    \label{pickup_hour}
\end{figure}

\begin{figure}
    \centering
    \begin{minipage}{.45\textwidth}
        \includegraphics[width=\linewidth]{trip_duration}
        \caption{Distribution of the trip duration, in seconds.}
        \label{trip_duration}
    \end{minipage}
    \hspace{0.05\textwidth}
   \begin{minipage}{.45\textwidth}
       \includegraphics[width=\linewidth]{log_trip_duration}
       \caption{Distribution of the logarithm of the trip duration, in seconds.}
       \label{log_trip_duration}
    \end{minipage}
\end{figure}

It is also important to make sure that the training and test sets are
independently and identically distributed. For this, the above distributions
have been compared with those from the test set. If the distribution from the
training and testing set significantly overlap, then it can be supposed that
the I.I.D. assumption is verified. As shown for example in figure
\ref{test_vs_train_long} and \ref{test_vs_train_lat}, it is indeed the case.

\begin{figure}
    \centering
    \begin{minipage}{.45\textwidth}
        \includegraphics[width=\linewidth]{test_vs_train_long}
        \caption{Drop-off longitude in the training and test set.}
        \label{test_vs_train_long}
    \end{minipage}
    \hspace{0.05\textwidth}
   \begin{minipage}{.45\textwidth}
       \includegraphics[width=\linewidth]{test_vs_train_lat}
       \caption{Drop-off latitude in the training and test set.}
       \label{test_vs_train_lat}
    \end{minipage}
\end{figure}

\section{Preprocessing}
Since the data is provided by Kaggle, it is already very clean. However, some
outliers can be found in the distributions of the GPS coordinates and the trip
durations. The mean value for the trip duration (961 seconds) and the standard
deviation (5247 seconds) have been calculated, and all trips which were longer
than the mean plus twice the standard deviation have been removed. Futhemore,
all trips which were located far outside of NYC were removed, as it would
probably the case for a realistic taxi service. The geographic bounds are
$[-74.03;-73.77]$ for the longitude and $[40.63;40.85]$ for the latitude.

To augment our  dataset, the weather data of NYC in 2016 has been downloaded
\footnote{from IEM Computed Hourly Precipitation Totals by Iowa State University
(\url{https://mesonet.agron.iastate.edu/request/asos/hourlyprecip.phtml?network=NY_ASOS})}
and the current volume of precipitation has been added to each trip. Next, the
GPS coordinates of the pick-up and drop-off locations have been used to calculate
following properties:
\begin{itemize}
    \item Distance between the two points curved along the surface of the
earth called, Vincenty's distance;
    \item Pythagorean distance between the two points;
    \item Manhattan distance;
    \item The angle between the two points in relation to the north pole, called
    bearing.
\end{itemize}
OpenStreetMap Route Machine (OSRM, \url{http://project-osrm.org/}) has been used
to add further information about the taxi ride. OSRM makes it possible to
download maps from all around the world and calculate the street distance
between two points as well as other information. Besides the street distance,
the number of turns to make before reaching the destination has been included,
as well as how many intersections one has to pass. Apart from the data provided
by Kaggle it was possible to use more trip records available on the website of
the NYC Taxi and Limousine Commission. However the files available are of the
size of a few gigabytes per month, which is why it was decided not to include
them.

Additionally to the data augmentation, each sample is shifted and scaled, so
that each column has zero mean and unit variance, as required for most of the
models that have been used. This functionality is easily provided by the SKlearn
library.

\section{Feature Selection}
In order to ensure that the model was not being fed useless features, a feature
selection elimination procedure was undertaken. The results are shown in the
above figure - the lower the ranking the better the feature is - by using the
Sklearn package recursive feature elimination (RFE) on top of a random forest
model. This function attempted to find a ranking of the features by doing a
prediction while removing one feature at the time. As expected the most
important features such as drop-off and pick-up locations as well as different
measures of distance were ranked as very important. In contrast, features that
did not contain much information such as pick-up year,
\textit{store\_and\_fwd\_flag}, \textit{vendor\_id} and \textit{pickup\_month},
were ranked as very low importance and hence were removed from the model.
Interestingly enough, the precipitation feature providing an idea of the weather
conditions, did not rank very low leading to believe that it did not have a
significant effect in the duration of the trips within New York and specifically
Manhattan, which is where most of the journeys were recorded.

On the other hand an analysis of feature importance using XGBoost in figure
\ref{information_gain} provides a different perspective on which features
provide the most information to the model. This information implies the relative
contribution of the corresponding feature to the model calculated - calculated
by taking the average information gain produced by each feature on each split of
the constructed trees. It is expected that the distance features add the most
significant amount of information, with exception of the Manhattan distance which
apparently adds little information - contradicting the RFE analysis.
Furthermore, the amount of intersections, day of the week and day of the year
add significantly more value to the model than previous analysis. This is a good
indication of the different amount of congestion at different times of week and
year, whereas in this analysis it seems that the coordinate locations add less
value, as they change less due to being majorly close by within the city.
Finally, note that some of the features within this information gain analysis
have been removed from the last one. However, by removing any more features
within this selection, the models error increased. Hence, this selection of
features was chosen as the final one.

\begin{figure}
    \centering
    \includegraphics[width=0.8\linewidth]{feature_importance_seaborn}
    \caption{Average information gain of each feature in the XGBoost model.}
    \label{information_gain}
\end{figure}

\section{Methods for Model Selection}
In order to assess the performance of each model to be tested and compare the
results, an error metric was selected. To remain consistent with the scoring of
the Kaggle competition, the same evaluation function was chosen: Root Mean
Squared Logarithmic Error (RMSLE). This allows to estimate the leaderboard
ranking of the results and since it is logarithmic, it ensures that the
prediction for short trips have the same weighting as the predictions for long
trips. The RMSLE is calculated as

\[RMSLE = \sqrt{\frac{1}{n}\sum_{i=1}^n(\log(p_i+1) - \log(a_i+1))^2}\]

Where $n$ is the total number of entries in the data set, $p_i$ is the predicted trip
duration and $a_i$ is the true trip duration.  The dummy regressor was implemented
to use as a benchmark since its predictions employ basic rules, to further
compare the regression models. It achieved a high RMSLE of 0.810 for the
prediction in seconds and 0.802 for the prediction in minutes.  Additionally to
predicting the trip duration in seconds through regression, each model was
designed to predict the trip duration in minutes through classification for a
more logical and interpretable estimate. In real life scenarios, it would be
unrealistic to predict a taxi journey to the exact second due to the dynamic
environment of the city.

\section{Results}
As previously mentioned, the results of each model will be presented in terms of
their RMSLE score and the predictions in minutes through classification. The
final optimised results of all models can observed and compared in table
\ref{results}. Each model was initially tested by running with the default
parameters and it became evident that neural networks, random forest and XGBoost
performed much better than the remaining models. Those three models were
extensively tested and optimised with all data available in appendix
\ref{results_appendix}. A grid search was still applied to the less performant
models but no further time was invested in optimising them since it was evident
they would not outperform the top three models and hence the lack of testing
results in the appendix.

\subsection{Extreme Gradient Boosting}
Since this problem uses a relatively large dataset, it was expected that Extreme
Gradient Boosting (XGBoost) would perform well. Using the default parameter
values and all of the available features, it produced a baseline score of 0.394.
After applying grid search to optimise the hyper-parameters (see appendix
\ref{results_appendix}) and removing features suggested by the feature selection
elimination, the RMSLE was reduced to 0.346, making XGBoost perform slightly
better than NN and becoming the best tested model. $l_1$ and $l_2$
regularisation terms were also introduced in an attempt to further improve the
prediction accuracy but regardless of the values used, no effect could be
observed. While training the model, an internal parameter controls which metric
is used to calculate the evaluation for validation data. Unfortunately, no
customised function can be passed so the RMSE was used but it is believed that
if this parameter was customisable, using RMSLE as an error metric would
marginally improve the results.


\subsection{Neural Networks}
Neural networks (NN) performed well in this data set with some hyper-parameters
tuning. At first, due to the nonlinearity of the data, models with up to 3
hidden layers and 512 nodes were experimented on, however, the results were far
from satisfactory. As one tried to reduce the complexity of the model, in which
the optimum point was found to be with three layers of 64 nodes - errors reduced
down to around 0.35 RMSLE. Following this, hyper-parameters grid search such as
the type of solver, activation function, batch size and amount of regularisation
were conducted which led to finding our optimal model. The Adam solver was the
most efficient which was expected as it is a good solver for large datasets,
with a relu activation function, some regularisation and small batch sizes of
200, which brought the error down to 0.346.

\subsection{Random Forest}
Random forest builds multiple decision trees and merges them together to get a
more accurate and stable prediction. This algorithm proved to be a powerful
algorithm for this dataset as it brings an extra aspect of randomness into the
model, as when it is growing the trees, it searches for the best feature among a
random subset of features instead of searching for the best feature while
splitting a node. This increases the diversity of the model which in this case
resulted in better results. The key hyper-parameters in this model were the
number of estimators (number of trees in the forest), the minimum number of
samples required to split a node and the minimum number of samples required to
be at a leaf node. The prediction error was always reduced by the number of
estimators of which it was increased up to the hardware memory capability - 125
estimators which led to an error of 0.36. On the other hand, the other
parameters were more finely tuned to reach the conclusion that the lower the
number of minimum samples split and minimum samples leaf, the higher the
overfitting likelihood within the model. Hence, by increasing these parameters,
an optimum error value was found to be at 0.346 thereby reaching one of the most
performant models within the trials conducted.

\subsection{Support Vector Machine}
The power of Support Vector Machine (SVM) is appealing, given its low
generalisation error. However, the time complexity of the training phase of the
SVM is proportional to the cube of the number of training examples, and as a
result it is impossible to train on our entire dataset, which is composed of
more than 1,400,000 samples. Therefore, a SVM regressor has been trained on a
random subset of the training set, which would contain up to 20,000 random
samples. Past this value, the training time is simply unbearable.

On the other hand, since the evaluation of a trained SVM model is fast, it was
possible to validate the model on a larger portion of the training set, thus
giving a strong estimation of the generalisation performance. It turns out that
by training a SVM on 20,000 samples, testing it on 100,000 samples and with a
suitable hyper-parameter optimisation, a SVM model achieves a RSMLE of 0.359. By
using more powerful hardware, it would certainly be possible to achieve a
better score by using more training samples, in order to reduce the variance of
the learned estimator, and thus its generalisation error.

\subsection{Stochastic Gradient Descent}
Using the Stochastic Gradient Descent regressor from sklearn, an RMSLE of 0.487
has been obtained without any hyper-parameter tuning. Using grid search however
it was possible to achieve an RMSLE of around 0.455. While the features are
scaled before running the model, it does not seem to get closer to the true
objective function. This may be due to the nature of GPS coordinates which are
difficult to linearise.

\subsection{Gaussian Process}
Just as the support vector machine, the Gaussian Process (GP) has a time
complexity of $O(n^3)$. Therefore, it was unbearable to train a GP model with
the whole dataset. After a grid search to find the optimal hyper-parameters, a
GP model has been trained with a rational quadratic kernel, and a noise
reduction parameter $\alpha=0.01$, on 1000 random samples and then validated on
100,000 other samples. The RSMLE was calculated to be 0.469, which is most
probably due to underfitting, due to the very limited number of training
samples.

\begin{table}[h!]
    \centering
    \begin{tabular}{ccc}
        Model & RMSLE regression & RMSLE classification\\
        \hline
        XGBoost & 0.346 & 0.286 \\
        Neural Network & 0.347 & 0.136 \\
        Random Forest & 0.350 & 0.312 \\
        SVM & 0.359 & NA \\
        SGD & 0.475 & 0.400 \\
        Gaussian Process & 0.512 & 0.500 \\
        Ridge Regression & 0.539 & NA \\
        Dummy Regressor & 0.810 & 0.802 \\
        AdaBoost & 0.990 & 0.800 \\
        \hline
    \end{tabular}
    \caption{Comparison of the RMLSE of all tried models.}
    \label{results}
\end{table}

\section{Conclusion}
As our first real team machine learning project, we tackled this introductory
Kaggle competition with a lot of curiosity. Having enriched the provided data
significantly with help from OSRM we managed to establish a decent model with
Extreme Gradient Boosting, which ranks us XX on the Kaggle competition
leaderboard. While we did improve the data set we could have gone further by
including the available files on the NYC TLC’s website the increase the number
of samples, just as using a sophisticated way of linearizing GPS coordinates
like one-hot encoding could have helped the improve the predictions. This could
be studied in a next Kaggle competition using this kind of data. For the time
being, we can be be happy with what we learned in the process and look forward
to our next challenges as machine learners.

\newpage
\footnotesize
\bibliographystyle{ieeetr}
\bibliography{report}

\appendix
\newpage
\section{Additional Figures}
\begin{figure}[h!]
    \centering
    \includegraphics[width=0.8\linewidth]{confusion}
    \caption{Confusion matrix of the extended dataset.}
    \label{confusion}
\end{figure}

\section{Detailled Grid Search results}
\label{results_appendix}

\end{document}
